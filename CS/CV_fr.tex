%-----------------------------------------------------------------------------------------------------------------------------------------------%
%	The MIT License (MIT)
%
%	Copyright (c) 2015 Jan Küster
%
%	Permission is hereby granted, free of charge, to any person obtaining a copy
%	of this software and associated documentation files (the "Software"), to deal
%	in the Software without restriction, including without limitation the rights
%	to use, copy, modify, merge, publish, distribute, sublicense, and/or sell
%	copies of the Software, and to permit persons to whom the Software is
%	furnished to do so, subject to the following conditions:
%	
%	THE SOFTWARE IS PROVIDED "AS IS", WITHOUT WARRANTY OF ANY KIND, EXPRESS OR
%	IMPLIED, INCLUDING BUT NOT LIMITED TO THE WARRANTIES OF MERCHANTABILITY,
%	FITNESS FOR A PARTICULAR PURPOSE AND NONINFRINGEMENT. IN NO EVENT SHALL THE
%	AUTHORS OR COPYRIGHT HOLDERS BE LIABLE FOR ANY CLAIM, DAMAGES OR OTHER
%	LIABILITY, WHETHER IN AN ACTION OF CONTRACT, TORT OR OTHERWISE, ARISING FROM,
%	OUT OF OR IN CONNECTION WITH THE SOFTWARE OR THE USE OR OTHER DEALINGS IN
%	THE SOFTWARE.
%	
%
%-----------------------------------------------------------------------------------------------------------------------------------------------%


%============================================================================%
%
%	DOCUMENT DEFINITION
%
%============================================================================%

%we use article class because we want to fully customize the page and don't use a cv template
\documentclass[10pt,A4]{article}	


%----------------------------------------------------------------------------------------
%	ENCODING
%----------------------------------------------------------------------------------------

%we use utf8 since we want to build from any machine
\usepackage[utf8]{inputenc}		

%----------------------------------------------------------------------------------------
%	LOGIC
%----------------------------------------------------------------------------------------

% provides \isempty test
\usepackage{xifthen}

%----------------------------------------------------------------------------------------
%	FONT
%----------------------------------------------------------------------------------------

% some tex-live fonts - choose your own

%\usepackage[defaultsans]{droidsans}
%\usepackage[default]{comfortaa}
%\usepackage{cmbright}
\usepackage[default]{raleway}
%\usepackage{fetamont}
%\usepackage[default]{gillius}
%\usepackage[light,math]{iwona}
%\usepackage[thin]{roboto} 

% set font default
\renewcommand*\familydefault{\sfdefault} 	
\usepackage[T1]{fontenc}

% more font size definitions
\usepackage{moresize}		


%----------------------------------------------------------------------------------------
%	PAGE LAYOUT  DEFINITIONS
%----------------------------------------------------------------------------------------

%debug page outer frames
%\usepackage{showframe}			


%define page styles using geometry
\usepackage[a4paper]{geometry}		

% for example, change the margins to 2 inches all round
\geometry{top=1.25cm, bottom=-.6cm, left=1.5cm, right=1.5cm} 	

%use customized header
\usepackage{fancyhdr}				
\pagestyle{fancy}

%less space between header and content
\setlength{\headheight}{-5pt}		


%customize entries left, center and right
%\lhead{}
%\chead{}
%\rhead{}


%indentation is zero
\setlength{\parindent}{0mm}

%----------------------------------------------------------------------------------------
%	TABLE /ARRAY DEFINITIONS
%---------------------------------------------------------------------------------------- 

%for layouting tables
\usepackage{multicol}			
\usepackage{multirow}

%extended aligning of tabular cells
\usepackage{array}

\newcolumntype{x}[1]{%
>{\raggedleft\hspace{0pt}}p{#1}}%


%----------------------------------------------------------------------------------------
%	GRAPHICS DEFINITIONS
%---------------------------------------------------------------------------------------- 

%for header image
\usepackage{graphicx}

%for floating figures
\usepackage{wrapfig}
\usepackage{float}
%\floatstyle{boxed} 
%\restylefloat{figure}

%for drawing graphics		
\usepackage{tikz}				
\usetikzlibrary{shapes, backgrounds,mindmap, trees}


%----------------------------------------------------------------------------------------
%	Color DEFINITIONS
%---------------------------------------------------------------------------------------- 

\usepackage{color}

%accent color
\definecolor{sectcol}{RGB}{0,0,180}

%dark background color
\definecolor{bgcol}{RGB}{110,110,110}

%light background / accent color
\definecolor{softcol}{RGB}{225,225,225}


%============================================================================%
%
%
%	DEFINITIONS
%
%
%============================================================================%

%----------------------------------------------------------------------------------------
% 	HEADER
%----------------------------------------------------------------------------------------

% remove top header line
\renewcommand{\headrulewidth}{0pt} 

%remove bottom header line
\renewcommand{\footrulewidth}{0pt}	  	

%remove pagenum
\renewcommand{\thepage}{}	

%remove section num		
\renewcommand{\thesection}{}			

%----------------------------------------------------------------------------------------
% 	ARROW GRAPHICS in Tikz
%----------------------------------------------------------------------------------------

% a six pointed arrow pointing to the left
\newcommand{\tzlarrow}{(0,0) -- (0.2,0) -- (0.3,0.2) -- (0.2,0.4) -- (0,0.4) -- (0.1,0.2) -- cycle;}	

% include the left arrow into a tikz picture
% param1: fill color
%
\newcommand{\larrow}[1]
{\begin{tikzpicture}[scale=0.58]
	 \filldraw[fill=#1!100,draw=#1!100!black]  \tzlarrow
 \end{tikzpicture}
}

% a six pointed arrow pointing to the right
\newcommand{\tzrarrow}{ (0,0.2) -- (0.1,0) -- (0.3,0) -- (0.2,0.2) -- (0.3,0.4) -- (0.1,0.4) -- cycle;}

% include the right arrow into a tikz picture
% param1: fill color
%
\newcommand{\rarrow}
{\begin{tikzpicture}[scale=0.7]
	\filldraw[fill=sectcol!100,draw=sectcol!100!black] \tzrarrow
 \end{tikzpicture}
}



%----------------------------------------------------------------------------------------
%	custom sections
%----------------------------------------------------------------------------------------

% create a coloured box with arrow and title as cv section headline
% param 1: section title
%
\newcommand{\cvsection}[1]
{
	\begin{center}
		\large\textcolor{sectcol}{\textbf{#1}}
	\end{center}
}

%create a coloured arrow with title as cv meta section section
% param 1: meta section title
%
\newcommand{\metasection}[2]
{
%\begin{tabular*}{1\textwidth}{r r}
\footnotesize{#2} \hspace*{\fill} \footnotesize{#1}\\[1pt]
%\end{tabular*}
}

%----------------------------------------------------------------------------------------
%	 CV EVENT
%----------------------------------------------------------------------------------------

% creates a stretched box as cv entry headline followed by some paragraphs about 
% the work you did
% param 1:	event time i.e. 2014 or 2011-2014 etc.
% param 2:	event name (what did you do?)
% param 3:	institution (where did you work / study)
% param 4:	list of paragraphs
%
\newcommand{\cvevent}[4]
{

\begin{tabular*}{1\textwidth}{p{13.6cm}  x{3.9cm}}
	\textbf{#2} - \textcolor{bgcol}{#3} &   \vspace{2.5pt}\textcolor{sectcol}{#1}
\end{tabular*}

\vspace{-8pt}
\textcolor{softcol}{\hrule}
\vspace{6pt}

	\foreach \desc in {#4}{
		$\cdot$ \desc\\[3pt]
	}
	
\vspace{3pt}
}

% creates a stretched box as 
\newcommand{\cveventmeta}[2]
{
	\mbox{\mystrut \hspace{87pt}\textit{#1}}\\
	#2
}

%----------------------------------------------------------------------------------------
% CUSTOM STRUT FOR EMPTY BOXES
%----------------------------------------- -----------------------------------------------
\newcommand{\mystrut}{\rule[-.3\baselineskip]{0pt}{\baselineskip}}

%----------------------------------------------------------------------------------------
% CUSTOM LOREM IPSUM
%----------------------------------------------------------------------------------------
\newcommand{\lorem}
{Lorem ipsum dolor sit amet, consectetur adipiscing elit. Donec a diam lectus.}



%============================================================================%
%
%
%
%	DOCUMENT CONTENT
%
%
%
%============================================================================%
\begin{document}


%use our custom fancy header definitions
\pagestyle{fancy}	





%----------------------------------------------------------------------------------------
%	HEADER IMAGE
%----------------------------------------------------------------------------------------

%\begin{figure}[H]
%\begin{flushright}
%	\includegraphics[trim= 320 130 460 210,clip,width=0.2\linewidth]{myfoto.jpg}	%trimming relative to image size!
%\end{flushright}
%\end{figure}


%---------------------------------------------------------------------------------------
%	TITLE HEADLINE
%----------------------------------------------------------------------------------------

\vspace{-8pt}
\begin{center}
    {\HUGE \textsc{Boudrouss Réda} \textcolor{sectcol}{\rule[-1mm]{1mm}{0.9cm}} \textsc{CV}}\\[2pt]
    {\small Ingénieur logiciel \textbullet{} Développeur Fullstack}
\end{center}

\vspace{6pt}

%----------------------------------------
%   MÉTADONNÉES
%----------------------------------------
\metasection{Paris, France}{}
\metasection{linkedin.com/in/rboudrouss}{}
\metasection{github.com/rboudrouss}{\textbf{Statut :} Étudiant en Master 2 Science et Technologie du Logiciel}
\metasection{contact@rboud.com}{\textbf{Langues :} Français (Natif), Anglais (Professionnel), Arabe (Conversationnel)}
\metasection{+33 (0)7 55 78 85 16}{\textbf{Activités :} Game Jam, Open Source, Associations étudiantes}

\vspace{-15pt}
\textcolor{softcol}{\hrule}

\normalsize

%----------------------------------------
%   RÉSUMÉ
%----------------------------------------
\cvsection{Résumé}
\textbf{Ingénieur logiciel} en formation, passionné par le \textbf{développement web}, l'\textbf{open source} et les \textbf{architectures distribuées}. J'ai conçu et maintenu des \textbf{solutions robustes} utilisées par une \textbf{communauté active}. Mon \textbf{engagement associatif} m'a permis de piloter des projets \textbf{collaboratifs} et de faciliter la \textbf{communication interdisciplinaire} dans un environnement universitaire exigeant.\\
Actuellement à la recherche d'une \textbf{alternance pour 2025--2026}, je suis disponible pour débuter \textbf{dès maintenant}.

%============================================================================%
%                            FORMATION                                     %
%============================================================================%

\cvsection{Formation}

\cvevent{2024 -- 2026}{Master Science et Technologie du Logiciel}{Sorbonne Université}{
    {Implémentation de structures de données avancées et optimisation des performances pour des problématiques complexes. Conception de composants et d'architectures asynchrones réactives et distribuées, avec vérification de sûreté},
    {Maîtrise de l'assurance qualité, conception guidée par les tests, analyse statique, contrats de spécification.},
    {Développement d'une interface web pour MOPSA : portage de l'analyseur statique OCaml/C/C++ en JavaScript /WebAssembly, avec un front-end React/TypeScript.}
}

\cvevent{2020 -- 2024}{Licence Informatique}{Sorbonne Université}{
    {Solide formation en algorithmique, architecture des ordinateurs, systèmes d'exploitation, réseaux, bases de données, programmation fonctionnelle et orientée objet.},
    {Réalisation de nombreux projets collaboratifs pour mettre en pratique les enseignements, dans divers langages (C, Java, OCaml, Python, TypeScript...).}
}


%---------------------------------------------------------------------------------------
%	EDUCATION SECTION
%--------------------------------------------------------------------------------------
\vspace{-15pt}
\cvsection{Expériences}

\cvevent{2021 -- Présent}{Employé Polyvalent}{Lidl France, Asnières-sur-Seine}{
    {Supervision des caisses, gestion des stocks et rotations de produits, traitement des réclamations.},
    {Encaissement, conseil client, mise en rayon et entretien des espaces de vente.}
}

\cvevent{2023 -- Présent}{Responsable Technique}{Association Play, Sorbonne Université}{
    {Conception et déploiement d'une infrastructure virtualisée : serveurs Docker, orchestration CI/CD, configuration réseau et gestion de Google Workspace (mails centralisés).},
    {Création et optimisation du site vitrine (9 000 visiteurs uniques en 2025) : architecture responsive, SEO on-page et suivi analytics.},
    {Développement d'une Web App "Glyph" (Next.js SSR) pour le suivi de quêtes : authentification sécurisée, architecture back-end et base de données performante pour hauts trafics.}
}


%\textcolor{softcol}{\hrule}

%
\cvevent{2024 - 2025}{Président}{Association Play Sorbonne Université}{
	{Coordination : Définir et coordonner la stratégie globale, superviser les projets. Fédérer l'équipe autour d'objectifs communs, suivre l'avancement des initiatives.},
	{Communication : Assurer la représentation de l'association auprès des services universitaires, des associations partenaires, des sponsors et des institutions extérieures. Centraliser et redistribuer les informations.}
}

\vspace{-15pt}
\cvsection{Compétences}

Techniques : TypeScript, React, Next.js, OCaml, Python, Java, SQL \\
Outils : Docker, Kubernetes, CI/CD (GitHub Actions, GitLab CI), Express, tRPC, Tailwind CSS, Google Workspace Admin, Git/GitHub \\
Méthodes : TDD \& Model-based Testing, analyse statique de code, Design by Contract, DevOps \& Infrastructure as Code \\
Langues \& soft skills : Français (natif), Anglais (C1), arabe (B2), gestion de projet collaboratif, communication \& résolution de problèmes complexes \\

%--------------------------------------------------------------------------------------------------
%	ARTIFICIAL FOOTER (fancy footer cannot exceed linewidth) 
%--------------------------------------------------------------------------------------------------

\null
\vspace*{\fill}
\hspace{-0.25\linewidth}\colorbox{white}{\makebox[1.5\linewidth][c]{\mystrut  \textnormal{\textcolor{sectcol}{linkedin.com/in/rboudrouss} $\cdot$ \textcolor{sectcol}{github.com/rboudrouss}}}}


%============================================================================%
%
%
%
%	DOCUMENT END
%
%
%
%============================================================================%
\end{document}